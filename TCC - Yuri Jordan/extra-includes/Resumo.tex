% Resumo
\begin{center}
	{\Large{\textbf{Modelo de análise de sentimento para dados extraídos de redes sociais}}}
\end{center}

\vspace{1cm}

\begin{flushright}
	Autor: Yuri Jordan de Melo Oliveira\\
	Orientador: Prof. Dr. Plácido Antônio de Souza Neto
\end{flushright}

\vspace{1cm}

\begin{center}
	\Large{\textsc{\textbf{Resumo}}}
\end{center}

\noindent Percebe-se que, atualmente, a capacidade técnica de coletar, analisar e utilizar informações sobre pensamentos e opiniões humanas, têm seu desenvolvimento impulsionado devido a propagação das redes sociais, em escala mundial e que, essas capacidades, vêm sendo aplicadas para entendimento e tomadas de decisões, nas diversas áreas da sociedade como saúde, educação, política e economia. Realiza-se uma pesquisa exploratória e desenvolve-se um modelo para análise de sentimento de dados textuais extraídos de redes sociais, a fim de saber se, o que os usuários dessas redes escrevem nelas, possuem um sentimento positivo, neutro ou negativo. Para isso, é necessário gerar um modelo para esse tipo de análise, implementar os algoritmos necessários para tal feito e um serviço web, com o intuito de expor esses algoritmos e aplicá-los em um estudo de caso, com alguns dos pré-candidatos à Presidência da República do Brasil, de 2018. Diante disso, foram coletadas e analisadas 2470 mensagens, sendo classificadas como negativas, neutras ou positivas e marcadas em um mapa, o local de onde vieram.

\noindent Palavras-chave: Análise de sentimento, Processamento de Linguagem Natural, Redes Sociais.




