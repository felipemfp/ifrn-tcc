% Resumo em l�ngua estrangeira (em ingl�s Abstract, em espanhol Resumen, em franc�s R�sum�)
\begin{center}
	{\Large{\textbf{Sentiment analysis model to extract data from social network}}}
\end{center}

\vspace{1cm}

\begin{flushright}
	Author: Yuri Jordan de Melo Oliveira\\
	Supervisor: Prof. Plácido Antônio de Souza Neto, Ph.D.
\end{flushright}

\vspace{1cm}

\begin{center}
	\Large{\textsc{\textbf{Abstract}}}
\end{center}

\noindent Is noticed that, currently, the technical ability to collect, analyse and use information about humans`s opinions and thoughts, has its development boosted because of the spread of the social networks, to a world scale, these capabilities, have been applied for the understanding and decision-making, in the various areas of society such as health, education, politics and economy. An exploratory is made and a model for the analysis of textual data extracted from social networks is developed, in order to know if, what the users of these networks write in them, has a positive, neutral or negative feeling. To achieve this, it is necessary to generate a model for this type of analysis, implement the necessary algorithms for this purpose and a web service, in order to expose these algorithms and applying them in a case study, with some of the pre-candidates for the presidency of the Republic of Brazil, in 2018. It is then done, an exploratory research about. Therefore, 2470 messages were collected and analyzed, being classified as negative, neutral or positive and marked on a map, the place from which they came.

\noindent Keywords:  Sentiment analysis, Natural Language Processing, Social network.



