% Cap�tulo 3
\chapter{Trabalhos Relacionados}

Este capítulo apresenta um total de oito trabalhos realizados, correlatos ao tema abordado neste documento. 

\cite{Farzindar:15} apontam que o grande volume de dados existentes em redes sociais e na Web, em geral, pode beneficiar diversas áreas, tais como: a indústria, a mídia, a saúde, a política, a segurança, e que através de acesso à Ferramentas de Processamento  de Linguagem Natural, que processam informações textuais em tempo real, as ações de governos podem ganhar uma maior efetividade com tomadas de decisões mais estratégicas e operacionais. 
Em seus trabalhos são feitas revisões acerca das métricas, existentes naquela época (2015), para avaliação de Processamento de Linguagem Natural (PLN) realizado em aplicações de rede social, mostram também esforços executados por companhias como: \textit{Association for Computational Linguistics} (tal como o método SemEval), pelo  \textit{National Institute of Standards and Technology} através da \textit{Text REtrieval Conference} (TREC) e \textit{Text Analysis Conference} (TAC) para acharem formas eficientes e eficazes para aplicação da PLN, além de discorrerem sobre o potencial que o uso dessa técnica pode ter nas décadas que virão, contextualizando com mudanças nas tecnologias móveis, computação em nuvem e redes sociais.

\cite{ThakoreSasi:15} em seu artigo publicado na International \textit{Neural Network Society} (INNS)  no \textit{Conference on Big Data}, mostram um novo processo de Análise de sentimento para conteúdo de rede social.
Neste trabalho, mensagens são extraídas do Twitter, automaticamente. Em seguida elas são imputadas em um processo baseado em ontologia que analisa e devolve essas mensagens que contenham valor semântico negativo, com o objetivo de demonstrar a insatisfação de consumidores ao recorrerem aos serviços postais dos Estados Unidos, Reino Unido e Canadá.

\cite{Rosa:15} em sua tese para obtenção de título de Doutor em Engenharia Elétrica, propõe um novo procedimento utilizado para cálculo de sentimentos e afetividade em redes sociais. Esse procedimento leva em consideração alguns critérios de informações contidas nessas redes, tais como,  características pessoais, idade, gênero etc.
O estudo tem como resultado a aplicação, de uma métrica criada em seu trabalho denominada  \textit{Brazillian Affective Metric} (AFM-Br), a qual, extrai emoções, em um estudo de caso de um sistema que recomenda músicas de acordo com o estado emocional de um usuário. 

\cite{Weiand:16} em seu artigo propõe um algoritmo para análise semântica das mensagens postadas por usuários do Twitter (tweets).
Na sua proposta é empregado o modelo probabilístico de classificação Naïve Bayes, a utilização das bibliotecas NLTK (\textit{Natural Language Toolkit}) e Scikit-Learn e um dataset com tweets estruturados e já classificados semanticamente, frutos do trabalho de Sanders (2011).
Seu algoritmo de análise de sentimento tem como objetivo ser aplicado em trabalhos relacionados à computação gráfica, assim como, possa ser integrado em outros sistemas de rede social, blogs etc.

\cite{GaoEtAl:17} relatam um estudo de análise de texto em redes sociais proposto pelo \textit{Beijing Municipal Institute of Urban Planning and Design}. O objetivo desse estudo é de auxiliar aos administradores responsáveis pelo planejamento urbano da cidade Pequim, a melhorarem suas capacidades de percepção social sobre os dados gerados pelas tecnologias existentes nesse ambiente.
A quantidade de dados que contêm informações espaço-temporais, geradas por pessoas comuns, aliadas à ferramentas de automação no processamento desses dados, facilita a elaboração das decisões de logística urbana a serem tomadas.
A aplicação do estudo se dá através de uma estrutura reutilizável (\textit{framework}) que contém um conjunto de algoritmos para tópicos de Processamento Natural de Linguagem, tais como, análise de sentimento, mineração de opinião e extração de informação de conteúdos gerados por usuários de redes sociais chinesas.

Obter informações de espaço e tempo a respeito de acontecimentos que impactam na sociedade como, endemias e desastres naturais são de extrema importância, pois potencializam a difusão de informações mais ampla e rapidamente, permitem uma tomada de decisão melhor planejada e o encurtamento do período de resposta, para resolução do(s) problema(s).
Existem estudos, que demonstram a repercussão que as redes sociais têm, servindo como sensores para detecção desses eventos. 
Embora a relevância de sua utilização possa ser observada, ainda existem adversidades desafiadoras a serem enfrentadas como, volume de dados instáveis, heterogeneidade desses dados e falta de densidade (informações vazias). Para tais desafios,  \cite{GaoEtAlInfluenza:18} apresentam um estudo, o qual visa identificar padrões acerca de eventos ocorridos, e aplicam essa abordagem para gerar mapas de ocorrências de influenza nos Estados Unidos, através de dados coletados do Twitter.

\cite{MarozzoeBessi:18} realizam um estudo de caso sobre o Referendo Constitucional de 4 de Dezembro de 2016 na Itália, no qual, apresentam uma metodologia que tem como propósito analisar o comportamento de usuários de redes sociais e o impacto que sites de notícias têm, durante campanhas políticas caracterizadas por rivalidade entre grupos políticos distintos.
A primeira parte do trabalho analisa como usuários do Twitter manifestaram suas intenções de voto sobre o Referendo Constitucional Italiano, semanas antes do dia da votação, assim como, busca um entendimento acerca da evolução das tendências de voto, ou seja, se as intenções de voto mudaram durante o período de análise.
Na segunda etapa do estudo a meta é de compreender a relevância que novos sites noticiários apresentam sobre o Referendo.

\cite{CaoEtAl:18} utilizam a IBM Watson Alchemy API (\textit{application program interface}), para realizar uma análise de sentimento de dados, extraídos em grande volume, do Twitter, com o objetivo de mapear quando e onde os usuários dessa rede social postam suas opiniões. O que é chamado por eles de \textit{public sentiment}.
O estudo é realizado na cidade de Massachusetts nos Estados Unidos, e conta com mais de 880 mil tweets gerados por mais de 26 mil usuários ativos, em um período de aproximadamente seis meses.
Em algumas de suas conclusões, os autores demonstram que o sentimento das mensagens postadas tende a ser mais positivo em locais públicos nos finais de semana, entretanto mais negativo em áreas comerciais de Segunda à Sexta.

Os trabalhos de \cite{Farzindar:15}, \cite{Rosa:15}, \cite{GaoEtAl:17}, \cite{MarozzoeBessi:18} e \cite{CaoEtAl:18} se assemelham, a este, em três quesitos: realiza coleta de dados em redes sociais, utiliza análise de sentimento, isso tudo, demonstrado em um estudo de caso.

Entretanto, os trabalhos de \cite{ThakoreSasi:15} e \cite{Weiand:16} possuem apenas duas, das três condições, estabelecidas para comparação, exceto utilização de estudo de caso. Enquanto \cite{GaoEtAlInfluenza:18} também, apresenta dois quesitos relacionados, porém, excetua-se no de realização de análise de sentimento.

A seguir, a Tabela 1, mostra a relação, existente, entre os trabalhos correlatos e este.


\begin{table}[!htb]
   \textsf{\caption{Relações entre este trabalho e seus correlatos.}}
   \centering
   \medskip

    \begin{tabular}{ |p{4cm}|p{3cm}|p{3cm}|p{3cm}|  }
     \hline
     \multicolumn{4}{|c|}{Relações entre este trabalho e seus correlatos} \\
     \hline
       \diagbox{Trabalhos}{Relações} & Realiza coleta de dados em redes sociais & Aplica Análise de Sentimento & Utiliza estudo de caso\\
     \hline
     \hline
      Natural language processing for social media \cite{Farzindar:15}   & X & X &   \\
     \hline
     
    \hline
       Ontology-based sentiment analysis process for social media content \cite{ThakoreSasi:15}& X & X &  X \\
    \hline
    
    \hline
      Análise de sentimentos e afetividade de textos extraídos das redes sociais \cite{Rosa:15}   & X & X &  X \\
    \hline
    
    \hline
      Análise de sentimentos do twitter com naïve bayes enltk \cite{Weiand:16}& X & X & \\
    \hline
    
    \hline
      Ontology-based social media analysis for urban planning \cite{GaoEtAl:17}   & X & X & X \\
    \hline
    
    \hline
      Mapping spatiotemporal patterns of events using social media \cite{GaoEtAlInfluenza:18}& X &  &  X \\
    \hline
    
    \hline
     Analyzing polarization of social media users and news sites during political campaigns \cite{MarozzoeBessi:18}  & X & X & X  \\
    \hline
    
    \hline
       Using twitter to better understand the spatiotemporal patterns of public sentiment: A case study in massachusetts, usa \cite{CaoEtAl:18} & X & X &  X \\
    \hline
    
    \hline
    \end{tabular}
    
\end{table}































