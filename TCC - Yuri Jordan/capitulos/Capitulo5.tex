% Cap�tulo 5
\chapter{Conclusão, Limitações e Trabalhos Futuros}

Neste trabalho, foram realizados uma pesquisa exploratória acerca da análise de sentimento, baseada em léxico, implementação de um serviço web onde os algoritmos são aplicados em um estudo de caso, com cinco pré-candidatos a presidência do Brasil de 2018, no qual dados textuais, são coletados do Twitter. Por último, os resultados obtidos com a análise são mostrados, com o intuito de saber se os comentários dos usuários destas redes sociais, a respeito dos pré-candidatos à presidência da República, foi positiva, neutra ou negativa.

Com isso, foi possível obter uma ferramenta de análise de sentimento, capaz de automatizar a avaliação de dados da rede social Twitter. O que foi demonstrado, nos capítulos dois e três, ser um objeto de estudo de interesse em pesquisas acadêmicas e trabalhos, aplicados, nas áreas da sociedade como saúde, entretenimento, segurança, mobilidade, entre outros.

Embora, o estudo de caso tenha sido efetuado, ocorrem algumas limitações de desenvolvimento. Tais limites são escopo de quantidade de redes sociais, pelas quais, os dados são coletados, pouco volume de dados geográficos, sobre os \textit{tweets}, limitação da análise de sentimento a apenas dados em língua portuguesa e serviço web REST, com estrutura mínima, ou seja, apenas apresenta as funcionalidades essenciais para execução do estudo de caso.

Espera-se que com a continuação deste projeto, seja possível unir o serviço web a um sistema de visualizações, com interfaces gráficas dinâmicas, em um \textit{dashboard}. Além de classificar os dados oriundos de mais idiomas, em mais sentimentos. Assim como implementar outros modos de análise de sentimento, como o \textit{Deep Learning}, e fazer um comparativo de eficiência, entre eles.






