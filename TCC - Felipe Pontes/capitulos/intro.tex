\chapter{Introduction}
\label{chap:intro}

More than ever we are overwhelmed by the amount of information has been created.
When we compare how much data has been created over the past years, we realize
it is already increasing exponentially.
Besides this quantitative evolution, nowadays
we have the most diverse kinds of data
(e.g. documents, tweets, pictures, videos, GIF, check-ins).

This phenomenon has been called \textit{Big Data} and represents an increasing field of study
for the time being. Therefore researchers all around the world
are analyzing and learning with these information we create everyday.
However the increasing amount of data is making analyses a way more difficult.
So we are investing in techniques and tools to tackle problems such as data mining, data cleaning,
data visualization, data classification, data exploration and so on.

Right in the middle of this scenario we may find data that comes along with latitude and longitude
(tweets and check-ins are good examples). We categorize those data as \textit{spatial data}.
Spatial data can be very insightful, for instance, a check-in at the airport by your sister
in the morning of your birthday, probably it means you will have a surprise.

% TODO: talk about how we explore spatial data (in map, Tableau etc) and how it take us to the problem definition

\section{Problem Definition}

The large size of spatial data make the analyst feel lost during the exploration.
There could be thousands of points in each neighborhood of a city. Analysts require to obtain
only few options (so-called "highlights") to act as a direction and be able to focus on.
In the perfect scenario, these options are not randomly chosen and represent what they showed to be interested.

In this work, we formulate the problem of "information highlighting using implicit feedback collected over time", i.e.,
highlight few spatial points based on implicit interests of the analysts in order to guide her
towards what she should concentrate on in consecutive iterations of the analysis process.

% TODO: "what to see next" question, talk about other environment 

\subsection{Case Study}

We discuss a real-world example to show the functionality of our approach in practice.

{\bf Example.} {\em Lucas is planning to spend few days in Paris, France. His appreciation of French culture makes him interested in new experiences in the city. He decides to rent a home-stay from Airbnb website\footnote{\it http://www.airbnb.com}. He likes to discover the city, hence he is open to any type of lodging in any region with an interest to stay in the city center. The website returns $4000$ different locations. As he has no other preferences, an exhaustive investigation needs scanning each location independently which is nearly infeasible. While he is scanning few first options, he shows interest in the region of ``Champ de Mars'' (near Eiffel Tower), but he forgets or doesn't feel necessary to click a point there. By discovering IDRs on his mouse moves over the home-stays in Paris, our system can quickly detect his interest in the region and short-list a small subset of locations (i.e., highlights) accordingly to be recommended to Lucas.}

% TODO add temporal analysis to it, remove IDRs (to complex to use now, maybe after model definition)

We follow the above example to describe how temporal analyses can be effectively
applied in the highlighting of interesting information based on previous experience
in the next sections.

\section{Objectives}

In this section, we define the general and specific objectives of our work.

\subsection{General Objectives}

\begin{itemize}
	\item Introduce a time-aware guidance approach for spatial data\ldots
	\item Elaborate how temporal analyses can be effectively applied in data exploration\ldots
	\item Present the results of our guidance approach\ldots
\end{itemize}

\subsection{Specific Objectives}

\begin{itemize}
	\item Present our proposed guidance approach\ldots
	\item Describe our data model used for temporal analyses\ldots
	\item Describe our concept of IDR used for collecting feedback\ldots
	\item TODO\ldots
\end{itemize}

\section{Organization}

The next sections is as follow: in the Chapter 2 we discuss the background of this work.
Chapter 3 defines the data model.
Chapter 4 presents how the feedback is collected during exploration.
Chapter 5 presents how temporal analysis is applied.
Chapter 6 presents how highlight interesting points in order to guide the user using collected feedback and results from temporal analysis.
Chapter 7 shows experiments and its results.
Chapter 8 presents some conclusions and future directions.
