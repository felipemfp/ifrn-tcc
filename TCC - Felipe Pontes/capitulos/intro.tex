\chapter{Introduction}
\label{chap:intro}

% TODO: improve with citations and goals
More than ever we are overwhelmed by the amount of information has been created.
When we compare how much data has been created over the past years, we realize
it is already increasing exponentially.
Besides this quantitative evolution, nowadays
we have the most diverse kinds of data
(e.g. documents, tweets, pictures, videos, GIF, check-ins).

This phenomenon has been called \textit{Big Data} and represents an increasing field of study
for the time being. Therefore researchers all around the world
are analysing and learning with the data we create everyday.
However the increasing amount of data is making analyses a way more difficult.
So we are investing in techniques and tools to tackle problems such as data mining, data cleaning,
data visualization, data classification, data exploration and so on.

Right in the middle of this scenario we may find data that comes along with a latitude and longitude
(tweets and check-ins are good examples). We categorize those data as \textit{spatial data}.
Spatial data can be very insightful, for instance, a check-in at the airport by your sister
in the morning of your birthday, probably it means you will have a surprise.
The problem comes when we have to analyse those data and due to its specificness it can
be difficult. % A valuable spatial data has a latitude and longitude (maybe an altitude) and more attributes
% about itself (e.g. a review or the tweet content).

GeoGuide is about making easy to the researcher explore, visualize and learn with spatial datasets.
In order to accomplish a better user experience, we provide a guidence approach based
on his preferences and explicit feedback.
GeoGuide continuously improves its guidance algorithm while
researcher is exploring the dataset. The framework captures implicit feedback
and try to understand what the user wants by analysing how the
researcher preferences has evolved over time.


\section{The problem}

TODO

\subsection{Use Case}

George is a software engineer with a very good taste for food. He likes to go to restaurants every month. George likes to try new restaurants and he already knows almost every place in his city, Parnamirim. So now George is looking for places in the nearby city, Natal. Before GeoGuide, George founds himself losing a lot of time choosing restaurant instead of actually going to them. And George changes his preferences with the seasons: another challenge for him during the Yelp exploration. For example, in the winter George prefers places close to his home or his work and in the summer, he prefers places which offers sea food.

By using GeoGuide during the past year to get to know almost every place in his city, when George starts to looking for places in the nearby city, GeoGuide already understands the behavior of George's preferences over the seasons and highlight suggestions of places most likelly to be interesting for him according to the moment of the year.

\section{Goals}

We iam for the this work to be...

\begin{itemize}
	\item Super duper goal 1\ldots
	\item Awesome goal 2\ldots
\end{itemize}

\subsection{Specific goals}

List of specific goals of this work:\ldots

\begin{itemize}
	\item Something 1\ldots
	\item Another 2\ldots
\end{itemize}

\section{Organization}

In the first half of this paper, we discuss...
