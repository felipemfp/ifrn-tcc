\chapter{Considerações Finais}
\label{chap:conclusao}

Neste trabalho é proposto um modelo espaço-temporal para identificação de padrões nas preferências coletados pela ferramenta GeoGuide enquanto um analista realiza a análise exploratória de dados espaciais. O modelo proposto mostra-se eficiente para ser utilizado a fim de aprimorar o uso da captura de feedback implícito para orientar o usuário nas próximas iterações através do método de destacamento de pontos geográficos.

\section{Contribuições}

O modelo propõe 2 novos conceitos: $i$. regiões densas interessantes e $ii$. análise temporal das preferências do usuário. As regiões densas interessantes (IDR) representam as preferências do analista no contexto espacial no determinado momento. Cada IDR possue também um perfil que representa as preferências do analista no contexto de domínio.

As regiões e seus perfis permitem a análise temporal das preferências do usuário. Os métodos para análise em ambos contextos espaciais e de domínio são investigados e apresentados com o propósito de identificar padrões nas iterações da análise exploratória.

\section{Trabalhos futuros}

Os dados gerados pelo modelo proposto podem futuramente serem explorados em diversos cenários. Por exemplo, os dados gerados podem ser utilizados para entender as preferências de um grupo de analistas a fim de potencializar o descobrindo do objetivo comum entre eles.

No que diz respeito aos ambientes de exploração de dados espaciais, os dados coletados podem serem combinados com algoritmos preditivos para responder questionamentos como ``onde o analista deve está interessado na próxima iteração?'' ou até mesmo ``será que o analista está interessado nesse apartamento com varanda?''.  