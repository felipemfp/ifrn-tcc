% Introdu��o
\chapter{Introdução ...}
\label{chap:introducao}

% A introdução é a parte inicial do texto e que possibilita uma visão geral de
% todo o trabalho, devendo constar a delimitação do assunto tratado, objetivos da
% pesquisa, motivação para o desenvolvimento da mesma e outros elementos
% necessários para situar o tema do trabalho.

% increasing volume of spatial data
% geoguide
% implicit feedback
% analyze feedback over time

More than ever we are overwhelmed by the amount of data we have been
created. When we compare how many data has been created over the past years, we realize our data is already increasing exponentialy.

Among those data we have the most diverse kinds of data such as pictures, texts, structured data, spatial data, temporal data and a lot more. Researchers all around the world are already analysing and learning with the data we generated.

However the increasing amount of data is making analyses a way more difficult. So we are investing in techniques and tools about data cleaning, data visualization, data classification and so on.

Spatial data is right in the middle of this scenario because its specificness. A valuable spatial data has a latitude and longitude and more information about it. For instance, a checkin at the airport by your sister in the morning of your birthday, probably it means you'll have a surprise.

GeoGuide is about making easy to the researcher explore, visualize and learn with big spatial datasets. In order to provide a better experience, we provide a guidence approach based on his preferences and explicit feedback.

The next step was capturing implicit feedback while researcher is exploring the dataset on our plataform. With those captured data, we can improve the guidence algorithm and, consequently, help the researcher to see what he may have lost in that huge amount of data.

Now we'll further, we'll analyse how the researcher preferences evolve over time by tracking his implicit and explicit feedback.

\section{Objetivos}

Nesta seção são definidos os objetivos gerais e específicos do trabalho.

\subsection{Objetivos Gerais}

\begin{itemize}
  \item Objetivo geral 1\ldots
  \item Objetivo geral 2\ldots  
\end{itemize}

\subsection{Objetivos Específicos} 

Lista de objetivos específicos do trabalho\ldots

\begin{itemize}
  \item Objetivo específico 1\ldots
  \item Objetivo específico 2\ldots  
\end{itemize}

\section{Metodologia}

Na metodologia é descrito o método de investigação e pesquisa para o
desenvolvimento e implementação do trabalho que está sendo proposto.

\section{Organização do trabalho}

Nesta seção deve ser apresentado como está organizado o trabalho, sendo
descrito, portanto, do que trata cada capítulo.
