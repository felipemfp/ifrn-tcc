\chapter{Data Model Definition}
\label{chap:model}

We consider two layers: spatial layer and feedback layer.

{\bf In the spatial layer:} each point in a dataset ($p \in \mathcal{P}$) is described using its coordinates (latitude and longitude) and also associated with a set of attributes ($dom(p)$). For instance, %TODO

{\bf In the feedback layer:} we have IDRs per iteration/session where implicid feedback is captured such mouse moves (or eye gaze). In the beginning, each IDRs is a group of raw points described using its coordinates (latitude and longitude) and a timestamp (the unix timestamp it was captured). These raw points once captured will enter the clustering (for now, ST-DBSCAN) phase to generate the IDR itself with a profile. The profile is built based on the spatial layer and it should represent a summary of its contained points from the spatial layer.

\begin{itemize}
	\item A profile has summary of its spatial points number attributes. For each number attribute in $dom(p)$, we calculate the average, median and standard deviation based on the points contained in the IDR.

	\item A profile has a word rank $R$ of the terms in the text attributes of its spatial points. For each text attribute in $dom(p)$, we evaluate the most used terms in order to create a word rank \cite{kumarAndKaur}.

	\item A profile has a map $M$ between the $<name, value>$ of categoricals attributes and its relevance in $dom(p)$.

	\item TODO: datetime attributes

	\item A profile has a meta property with values such the count of points in the IDR.
\end{itemize}