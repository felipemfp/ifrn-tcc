\chapter{Data Model Definition}
\label{chap:model}


We consider two layers: spatial layer and feedback layer.

{\bf In the spatial layer:} each point in a dataset ($p \in \mathcal{P}$) is described using its coordinates (latitude and longitude) and also associated with a set of attributes ($dom(p)$). For instance, %TODO

{\bf In the feedback layer:} we have IDRs per iteration/session where implicid feedback is captured such mouse moves (or eye gaze). In the beginning, each IDRs is a group of raw points described using its coordinates (latitude and longitude) and a timestamp (the unix timestamp it was captured). These raw points once captured will enter the clustering (for now, ST-DBSCAN) phase to generate the IDR itself with a profile. The profile is built based on the spatial layer and it should represent a summary of its contained spatial points. 

\begin{itemize}
\item A profile has median of its spatial points number attributes (for each number attribute in $dom(p)$, it has a median for the points in the IDR).

\item A profile has a rank $R$ of the terms in the text attributes of its spatial points. %TODO

\item A profile has a feedback vector $F$ which contains a map between $<name, value>$ and relevance of a attribute in $dom(p)$. The feedback vector in the first iteration is filled with 0 and for each next iterations it will be updated by incremeting the relevance of an $<name, value>$ by a arbitrary value ($\sigma$) according with its presence in the IDR and normalizing it using Softmax. %TODO?
\end{itemize}