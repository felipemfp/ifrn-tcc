\chapter{Contextualização}
\label{chap:contextualizacao}

Este capítulo dá uma visão geral sobre os trabalhos relacionados sobre exploração de feedback, métodos de destacamento de informações e aplicação de análises temporais. Também apresentamos o sistema que estamos estendendo neste trabalho.

\section{Trabalhos Relacionados}

A literatura na análise de dados espaciais possue um foco no eficiência das interações exploratórias. A abordagem comum é projetar índeces pre-processados, o quais permitem a consulta eficiente de dados espaciais \cite{lins2013nanocubes}. No entanto, também devemos direcionar nossa atenção para o {\em valor} dos dados espaciais, porque é muito comum encontrar um analista se perdendo numa enorme quantidade de pontos geográficos. Para solucionar esse problemas, ambientes de visuação, como, por exemplo, Tableau\footnote{\it http://www.tableau.com}, Exhibit\footnote{\it http://www.simile-widgets.org/exhibit/}, Spotfire\footnote{\it  http://spotfire.tibco.com}, oferecem funcionalidades para manipular os dados como filtros, consultas agregadores entre outras.

\subsection{Exploração de Feedback}

Nosso modelo espaço-temporal aprimora o processo de análise de dados espaciais destacando subconjuntos de pontos geográficos com base no feedback coletado durante a exploração do analista. Na literatura, há várias trabalhos sobre exploração de feedback para orientar o analista nas futuras iterações da análise (por exemplo, \citeonline{boley2013one}). A abordagem comum é uma metodologia {\em top-$k$} para reduzir os escopo de consulta baseado no feedback explícito e recomendar um pequeno subconjunto de resultados interessantes de tamanho $k$. Uma clara distinção do nosso trabalho é que não buscamos reduzir o escopo, mas alavancar o conjunto de dados com resultados potencialmente interessantes que o analista talvez não tenha notado devido ao enorme volume de dados espaciais. Ao passo que as escolhas do analista são limitadas por $k$ em algoritmos de {\em top-$k$} processamento, oferecemos a liberdade de escolha ao mesmo tempo que pontos geográficos vão sendo destacados com base nas novas escolhas do analista.

\subsection{Métodos de Destacamento de Informação}

\todo{TALK ABOUT EACH ONE}

Há trabalhos na literatura sobre métodos de destacamento de informações, por exemplo: \citeonline{Liang2010,Robinson2011,wongsuphasawat2016voyager,willett2007scented}. Entretanto todos esses métodos são {\em objetivos} e não são aplicáveis para o contexto de orientação espacial onde o feedback do usuário é envolvido. Em termos de recomendação, algumas abordagens focam na dimensão espacial \cite{Bao2015,Levandoski:2012} enquanto o contexto e a diversidade do resultado é deixado de lado.

\subsection{Aplicações de Análise Temporal}

Existem várias instâncias na literatura que combinam análise temporal com dados espaciais, como \citeonline{baculo2017,balahadia2017,chidean2018,ghahramani2018,kamath2013,lopestexeira2018,ma2017,mijovic2016,tomoki2010,nara2007,zhan2017,zheng2018}. Essas aplicações de análise temporal são em contextos específicos, os quais não envolve feedback do usuário, mas representam como análise temporal pode ser perspicaz e proveitosa.

\citeonline{baculo2017} e \citeonline{balahadia2017} fazem uso de dados públicos de Manila, capital das Filipinas, combinando dados espaciais, análise temporal e modelos preditivos e mostrando resultados que podem ser utilizados para preparação de um plano de gestão pública eficaz. \citeonline{ma2017} e \citeonline{zheng2018} também fazem análises realistas de como eventos, como protestos, impactam na trajetórias de taxis, cujos resultados podem auxiliar no controle de tráfego urbano e nos planos de serviços de transporte da cidade. Ambos realizam ricas análises, as quais vamos usuar como inspiração.

\citeonline{chidean2018} apresenta como detectar padrões espaço-temporal no contexto do uso de energia eólica na Península Ibérica usando o algoritmo {\em Second-Order Data-Coupled Clustering}. Apesar do estudo detalhado, esse trabalho não contempla um contexto exploratório.

\citeonline{ghahramani2018}, \citeonline{lopestexeira2018} e \citeonline{zhan2017} demostram como análise temporal pode ser aplicado no contexto geográfico. \citeonline{zhan2017} vai além gerando uma árvore de clusterização hierárquica. Independentemente dos métodos e resultados, esses trabalhos não contribuem para o assunto em questão.

\citeonline{kamath2013} propõe uma abordagem de {\em reinforcement learning} para prever eventos (adoção de {\em memes}) num contexto espaço-temporal.

\citeonline{nara2007} introduz um modelo de visualização 3D para dados espaço-temporais que ajuda a analisar qualitativa e quantitativamente os padrões e tendências espaço-temporais. Vamos fazer uso dessa abordagem para demostrar os dados coletados no Capítulo \ref{chap:coletando}.

\section{GeoGuide}

\begin{figure}[t]
	\centering
	\includegraphics[width=\columnwidth]{imagens/framework}
	\caption{GeoGuide Framework}
	\label{fig:framework}
	\vspace{-10pt}
\end{figure}

GeoGuide  \cite{omidvarTehrani2017} é um ambiente de visualização de dados espaciais que coleta as preferências do usuário durante a exploração para destacar subconjuntos de pontos geográficos que podem ser interessantes ao analista. Figura \ref{fig:framework} ilustra os principais componentes da arquitetura do GeoGuide que apresentaremos nas próximos subseções.

\subsection{Preprocessamento}

GeoGuide precisa de um passo de preprocessamento para criar o índice que serão usados durante a fase de destacamento. O índice é uma tabela comparativa entre todos os pontos usando duas métricas de qualidade: relevância e diversidade.

\subsubsection{Relevância}

Relevância representa o quão similar é o ponto $a$ com o ponto $b$ num conjunto de dados. GeoGuide use a relevância para destacar pontos similares ao feedback do analista.

\subsubsection{Diversidade}

Diversade representa quão distante o ponto $a$ está localizado do ponto $b$. GeoGuide usa a diversidade para permitir ao analista explorar diferentes regiões, mas ainda assim trabalhar com pontos relevantes ao seu interesse.

\subsection{Preferências do Usuário}

Para coletar as preferências do usuário, GeoGuide usa ambos feedback implícitos e explícitos. Feedback explícito é quando o usuário está analisando os atributos de um ponto, por exemplo a descrição de uma casa no Airbnb, e explicitamente pede para explorar pontos similares ao selecionado. Feedback implícito é coletado atráves do rastreio dos movimentos do mouse e métricas como  ``quanto tempo o usuário passou analisando o perfil de um ponto''.

\subsection{Destacamento de Dados Espaciais}

GeoGuide combina o índice preprocessado e o feedback coletado para destacar subconjuntos de dados espaciais de acordo com as preferências do analista. O processo de destacamento provou ser eficiente em termos de ``quantos passos o analista leva até completar a tarefa de encontrar um ponto com um determinado perfil''. Usando GeoGuide, analistas foram capazes de completar a tarefa usando em média 10.7 passos, enquanto que usando Tableau, foram necessários 43 passos.

\vspace{25pt}

\noindent Neste trabalho, potencializamos GeoGuide à dois novos conceitos: $i$. regiões densas interessantes e $ii$. entendendo como as preferências do analista evoluem conforme o tempo.