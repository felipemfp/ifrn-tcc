% Pre-ambulo
\documentclass[a4paper, 12pt]{abnt}

%% Pacotes para texto em Ingl�s
% \usepackage[brazil]{babel}
% \usepackage[T1]{fontenc}
% \usepackage[latin1]{inputenc}
 
%% Pacotes para texto em Portugues
\usepackage[brazil]{babel}
\usepackage[utf8]{inputenc}
\usepackage[T1]{fontenc}
 
\usepackage{dsfont}
\usepackage{amssymb,amsmath}
\usepackage{multirow}
\usepackage[alf]{abntcite}
\usepackage[pdftex]{color, graphicx}
\usepackage{colortbl} 
\usepackage{url}
\usepackage{abnt-alf}
\usepackage{abntcite}
\usepackage{algorithm}
\usepackage{algorithmic}
%\usepackage{alg} 
%\usepackage{hyperref}


% Redefinicao de instrucoes
\floatname{algorithm}{Algoritmo}
\renewcommand{\algorithmicrequire}{\textbf{Entrada:}}
\renewcommand{\algorithmicensure}{\textbf{Saida:}}
\renewcommand{\algorithmicend}{\textbf{fim}}
\renewcommand{\algorithmicif}{\textbf{se}}
\renewcommand{\algorithmicthen}{\textbf{entao}}
\renewcommand{\algorithmicelse}{\textbf{senaoo}}
\renewcommand{\algorithmicfor}{\textbf{para}}
\renewcommand{\algorithmicforall}{\textbf{para todo}}
\renewcommand{\algorithmicdo}{\textbf{faca}}
\renewcommand{\algorithmicwhile}{\textbf{enquanto}}
\renewcommand{\algorithmicloop}{\textbf{loop}}
\renewcommand{\algorithmicrepeat}{\textbf{repetir}}
\renewcommand{\algorithmicuntil}{\textbf{ate que}}
\renewcommand{\algorithmiccomment}[1]{\% #1}


% Definicao da lista de simbolos
% \simb[entrada na lista de simbolos]{simbolo}:
% Escreve o simbolo no texto e uma entrada na lista de simbolos.
% Se o parametro opcional e omitido, usa-se o parametro obrigatorio.
\newcommand{\simb}[2][]
{%
	\ifthenelse{\equal{#1}{}}
	{\addcontentsline{los}{simbolo}{#2}}
	{\addcontentsline{los}{simbolo}{#1}}#2
}
% Para aceitar comandos com @ (at) no nome
\makeatletter 
% \listadesimbolos: comando que imprime a lista de simbolos
\newcommand{\listadesimbolos}
{
	\pretextualchapter{Lista de s��mbolos}
	{\setlength{\parindent}{0cm}
	\@starttoc{los}}
}
% Como a entrada sera impressa
\newcommand\l@simbolo[2]{\par #1}
\makeatother


% Definicao da lista de abreviaturas e siglas
% \abrv[entrada na lista de simbolos]{abreviatura}:
% Escreve a sigla/abreviatura no texto e uma entrada na lista de abreviaturas e siglas.
% Se o parametro opcional e omitido, usa-se o parametro obrigatorio.
\newcommand{\abrv}[2][]
{%
	\ifthenelse{\equal{#1}{}}
	{\addcontentsline{loab}{abreviatura}{#2}}
	{\addcontentsline{loab}{abreviatura}{#1}}#2
}
% Para aceitar comandos com @ (at) no nome
\makeatletter 
% \listadeabreviaturas: comando que imprime a lista de abreviaturas e siglas
\newcommand{\listadeabreviaturas}
{
	\pretextualchapter{Lista de abreviaturas e siglas}
	{\setlength{\parindent}{0cm}
	\@starttoc{loab}}
}
% Como a entrada sera impressa
\newcommand\l@abreviatura[2]{\par #1}
\makeatother


% \listofalgorithms: comando que imprime a lista de algoritmos
\renewcommand{\listalgorithmname}{Lista de algoritmos}


\newcommand{\myThesis}{An Spatial-Temporal Model to Explore Interesting Dense Regions over Time}
\newcommand{\myName}{Felipe Mateus Freire Pontes}
\newcommand{\mySupervisorName}{Dr. Plácido Antônio de Souza Neto}

\newcommand{\todo}[1]{}


% Hifeniza������o de palavras feita de forma incorreta pelo LaTeX
\hyphenation{PYTHON ou-tros}


% Inicio do documento
\begin{document}

\frenchspacing

% Capa (arquivo Includes/Capa.tex)
% Capa
% Prote��o externa do trabalho e sobre a qual se imprimem as informa��es indispens�veis 
% � sua identifica��o.

% Especifica��o da capa
\begin{titlepage}
	\begin{center}
		
		  
		\begin{minipage}{11.15cm}
			\begin{center}
				\begin{espacosimples}
					{\small \ \\
                       \textsc{Instituto Federal do Rio Grande do Norte}
                       \\
							  \textsc{Campus Natal - Central}					\\
							  \textsc{Diretoria de Gestão e Tecnologia da Informação}	   
							  \\
							  \textsc{Tecnologia em Análise e Desenvolvimento de Sistemas}}   	
                       \\
				\end{espacosimples}
			\end{center}
		\end{minipage}

			
		\vspace{6cm}
						
		% T�tulo do trabalho
		{\setlength{\baselineskip}%
		{1.3\baselineskip}
		{\LARGE \textbf{\myThesis}}\par}
			
		\vspace{3cm}
			
		% Nome do aluno (autor)
		{\large \textbf{\myName}}
						
		\vspace{6cm}
		
		% Local da institui��o onde o trabalho deve ser apresentado e ano de entrega do mesmo
		Natal-RN\\\myDeriveryDate
	\end{center}
\end{titlepage}

% Folha de rosto (arquivo extra-includes/FolhaRosto.tex)
% Folha de rosto
% Cont�m os elementos essenciais � identificação do trabalho.

% T�tulo, nome do aluno e respectivo orientador e filiação
% \titulo{\Large{\myThesis}}
% \autor{\myName}
% \orientador[Orientador]{\par \mySupervisorName}
% \instituicao
% {
%    TADS -- Curso de Tecnologia em Análise e Desenvolvimento de
%    Sistemas\par 
%    DIATINF -- Diretoria Acadêmica de gestão e Tecnologia da Informação\par 
%    CNAT -- Campus Natal - Central\par 
%    IFRN -- Instituto Federal do Rio Grande do Norte }
	
% % Natureza do trabalho (não deve ser modificada)
% \comentario
% {
% 	Trabalho de conclusão de curso de graduação do curso de Tecnologia e Análise em
% 	Desenvolvimento de Sistemas da Diretoria de Gestão e Tecnologia de Informação
% 	do Instituto Federal do Rio Grande do Norte como requisito parcial para a
% 	obtenção do grau de Tecnologo em Análise e Desenvolvimento de
% 	Sistemas.\bigskip\\
%    \textit{Linha de pesquisa}:\\\myLineOfResearch
% }
		
% % Local e data
% \local{Natal-RN}
% \data{\myDeriveryDate}
	
% \folhaderosto

\begin{folhaderosto}
   %\imprimirfolhaderosto{}
   
     \begin{center}
       \MakeUppercase{\imprimirautor}
       \vspace*{\fill}\vspace*{\fill}
       \vspace*{\fill}\vspace*{\fill}
       \vspace*{\fill}\vspace*{\fill}
       \vspace*{\fill}\vspace*{\fill}
       \vspace*{\fill}\vspace*{\fill}
       \vspace*{\fill}\vspace*{\fill}
       \vspace*{\fill}\vspace*{\fill}
       \begin{center}
         \bfseries\MakeUppercase{\imprimirtitulo}
       \end{center}
      %  \vspace*{\fill}
     \end{center}
   
     %\vspace*{\fill}
   
     \hspace*{\fill}
     \begin{minipage}{.5\textwidth}
       \SingleSpace
       \imprimirpreambulo
       \\[\baselineskip]
       Orientador: Dr. \imprimirorientador
     \end{minipage}
   
     \vspace*{\fill}\vspace*{\fill}
     \vspace*{\fill}\vspace*{\fill}
     \vspace*{\fill}\vspace*{\fill}
     \vspace*{\fill}\vspace*{\fill}
   
     \begin{center}
         \imprimirlocal{}\\
         \imprimirdata
         \par
     \end{center}
   
   \end{folhaderosto}

% Folha de aprovacao (arquivo extra-includes/FolhaAprovacao.tex)
% % Folha de aprova��o
% \begin{folhadeaprovacao}
% 	\setlength{\ABNTsignthickness}{0.4pt}
% 	\setlength{\ABNTsignwidth}{10cm}

% 	\noindent
% 	Trabalho de Conclusão de Curso de Graduação sob o título
% 	\textit{\myThesis} apresentado por \myName{} e aceito pela Diretoria
% 	de Gestão e Tecnologia da Informação do Instituto Federal do Rio Grande do
% 	Norte, sendo aprovado por todos os membros da banca examinadora abaixo especificada:

% 	% Membros da banca examinadora e respectivas filia��es
% 	\assinatura
% 	{
% 	{\mySupervisorName}   			                  \\
% 	{\small Presidente}											          \smallskip\\
% 	{\footnotesize
% 	DIATINF -- Diretoria Acadêmica de Gestão e Tecnologia da Informação		   \\
% 	IFRN -- Instituto Federal do Rio Grande do Norte
% 	}
% 	}

% 	\assinatura
% 	{
% 	\myFirstExaminerName   			                  \\
% 	{\small Examinador}											          \smallskip\\
% 	{\footnotesize
% 	DIATINF -- Diretoria Acadêmica de Gestão e Tecnologia da Informação		   \\
% 	IFRN -- Instituto Federal do Rio Grande do Norte
% 	}
% 	}

% 	\assinatura
% 	{
% 	\mySecondExaminerName   			                  \\
% 	{\small Examinador}											          \smallskip\\
% 	{\footnotesize
% 	DIATINF -- Diretoria Acadêmica de Gestão e Tecnologia da Informação		   \\
% 	IFRN -- Instituto Federal do Rio Grande do Norte
% 	}
% 	}

% 	\vfill

% 	\begin{center}
% 		Natal-RN, \myDefenseDate.
% 	\end{center}
% \end{folhadeaprovacao}

\begin{folhadeaprovacao}
	\OnehalfSpacing
  
	\begin{center}
	  \MakeUppercase{\imprimirautor}
  
	  \vspace*{\fill}\vspace*{\fill}
	  \begin{center}
		\MakeUppercase{\imprimirtitulo}
	  \end{center}
	  \vspace*{\fill}
	\end{center}
  
	{
	  \hspace*{\fill}
	  \begin{minipage}{.5\textwidth}%
		\SingleSpacing
		\imprimirpreambulo%
	  \end{minipage}%
	}
	  \vspace*{\fill}
	  \vspace*{\fill}
	  \vspace*{\fill}
  
  Trabalho de Conclusão de Curso apresentado e aprovado em \_\_\_/\_\_\_/\_\_\_\_, pela seguinte Banca Examinadora:
  
	\centering
	  \begin{center}%
		BANCA EXAMINADORA
	  \end{center}%
  
	  \assinatura{
		\imprimirorientador, D.r -- Presidente\\
		Instituto Federal de Educação, Ciência e Tecnologia do Rio Grande do Norte
	  }
  
	  \assinatura{
		\examinadorA, D.r -- Examinador\\
		Instituto Federal de Educação, Ciência e Tecnologia do Rio Grande do Norte
	  }
  
	  \assinatura{
		\examinadorB, D.ra -- Examinador\\
		Instituto Federal de Educação, Ciência e Tecnologia do Rio Grande do Norte
	  }
  
	\vspace*{\fill}
	\begin{center}
	  \begin{SingleSpacing}
		\imprimirlocal{}\\
		\imprimirdata
	  \end{SingleSpacing}
	\end{center}
  
  \end{folhadeaprovacao}
  %---

% Dedicatoria (arquivo extra-includes/Dedicatoria.tex)
% Dedicat�ria


% \vspace{15cm}
% \begin{flushright}
	
% \end{flushright}


\begin{dedicatoria}
	\vspace*{\fill}
	\begingroup
	\leftskip=4cm
	\noindent%
	Aos meus pais que nunca duvidaram de mim.
	\par
	\endgroup
\end{dedicatoria}

% Agradecimentos (arquivo extra-includes/Agradecimentos.tex)
% Agradecimentos

\chapter*{Agradecimentos}

Ao meu orientador \mySupervisorName, por...

À Behrooz Omidvar-Tehrani, por...

Ao meu amigo Francisco Bento da Silva Júnior, por...

% Epigrafe (arquivo extra-includes/Epigrafe.tex)
% Ep�grafe (cita��o seguida de indica��o de autoria)

\chapter*{}
\vspace{15cm}
\begin{flushright}
	\textit
	{
		A coisa mais autêntica sobre nós é nossa capacidade de criar, de superar, de suportar, de transformar, de amar e de sermos maiores que nosso sofrimento.
	}\medskip\\ 
	Ben Okri
\end{flushright}

% Resumo em l���ngua vernacula (arquivo extra-includes/Resumo.tex)
% Resumo
% \begin{center}
% 	{\Large{\textbf{\myThesis}}}
% \end{center}

% \vspace{1cm}

% \begin{flushright}
% 	Autor: \myName\\
% 	Orientador: \mySupervisorName
% \end{flushright}

% \vspace{1cm}

% \begin{center}
% 	\Large{\textsc{\textbf{Resumo}}}
% \end{center}

% \noindent 

% \noindent\textit{Palavras-chave}: 

% \setlength{\absparsep}{18pt} % ajusta o espaçamento dos parágrafos do resumo
\begin{resumo}
	Este trabalho propõe um modelo espaço-temporal para exploração de preferências do usuário no cenário da análise exploratória de dados espaciais. A partir da observação das dificuldades enfrentadas por analistas durante a análise exploratória de grandes conjuntos de dados, verificou-se a possibilidade de combinar métodos de exploração de preferências do usuário com análise temporal, o que resultou no modelo de dados proposto neste trabalho. Para tanto, aprimorou-se a ferramenta GeoGuide, um ambiente de exploração de dados espaciais alinhado com métodos de destacamento de informações com base nas preferências do usuário, para coletar dados não somente no contexto espacial, mas também no contexto de domínio. Este trabalho detalha o processo para capturar de maneira transparente as preferências do usuário com o objetivo de viabilizar a aplicação do modelo proposto. Também apresenta como os dados coletados podem ser analisados temporalmente a fim de encontrar padrões nas preferências do usuário. Por fim, espera-se que o modelo proposto seja explorado, adaptado e aplicado além do cenário apresentado.

  %\vspace{\onelineskip}

  \noindent
  {Palavras-chave}: Dados Espaciais. Análise Temporal. Preferência do Usuário.
\end{resumo}

% Abstract, resumo em l���ngua estrangeira (arquivo Include/Abstract.tex)
% Resumo em l�ngua estrangeira (em ingl�s Abstract, em espanhol Resumen, em franc�s R�sum�)
\begin{center}
	{\Large{\textbf{\myThesisEnglish}}}
\end{center}

\vspace{1cm}

\begin{flushright}
	Author: \myName\\
	Supervisor: \mySupervisorName
\end{flushright}

\vspace{1cm}

\begin{center}
	\Large{\textsc{\textbf{Abstract}}}
\end{center}

\noindent This work proposes a spatial-temporal model to exploit user preferences in the scenario of an exploratory analysis of spatial data. From the observation of the difficulties faced by analysts during the exploratory analysis of large datasets, the possibility of combining methods of exploring user preferences with time analysis was verified, which resulted in the data model proposed in this work. Therefore, the GeoGuide tool, a spatial data exploration environment which provides a guidance approach using information highlighting methods based on user preferences, has been improved to collect data not only in the spatial context but also in the domain context. This work details the process to transparently capture user preferences with the objective of making the proposed model feasible. It also displays how the data collected can be analyzed temporally in order to find patterns in user preferences. With this work, it is expected that the proposed model will be explored, adapted and applied beyond the presented scenario.

\noindent\textit{Keywords}: Spatial data, Temporal Analysis, User Feedback.

% Lista de figuras
\listoffigures

% Lista de tabelas
\listoftables

% Lista de abreviaturas e siglas
\listadeabreviaturas

% Lista de símbolos
%	\listadesimbolos

% Lista de algoritmos (se houver)
% Devem ser inclu���dos os pacotes algorithm e algorithmic
% \listofalgorithms

% Sum���rio
\sumario

% Parte central do trabalho, englobando os cap��tulos que constituem o mesmo
% Os referidos cap��tulos devem ser organizados dentro do diret��rio "Cap��tulos"

% Capitulo 1: Introdu����o (arquivo Includes/Introducao.tex)
\chapter{Introdução}
\label{chap:intro}

Mais do que nunca estamos sobrecarregados com a quantidade de dados que criamos a cada dia. Quando comparamos quanto de informação vem sendo gerada nos ultimos anos, percebemos que está aumentando significamente. Além dessa evolução quantitativa, hoje temos os mais diversos tipos de informação, por exemplo: documentos, tuítes, fotos, vídeos, \textit{GIFs}, \textit{check-ins} entre vários outros.

Esse fenômeno vem sido chamado de \textit{Big Data} e representa uma crescente área de estudo atualmente. Como consequência, pesquisadores estão analisando e aprendendo com essas informações geradas, entretanto o crescimento contínuo da quantidade de dados dificulta as análises. Portanto pessoas estão investindo em novas técnicas e ferramentas para romper desafios como mineração de dados, {\em data cleaning}, visualização de dados, classificação de dados, exploração de dados e muito mais.

Um tipo comum de dados é o que chamamos de dado espacial, o qual a informação possui atributos geográficos como latitude e longitudade (por exemplo: tuítes, avaliação de restaurantes, {\em check-ins} em estabelecimentos). Dados espaciais podem ser muito significativo, por exemplo, um {\em check-in} no aeroporto por sua irmã na manhã do seu aniversário, provavelmente significa que você terá uma surpresa.

Cada registro em dados espaciais representa uma atividade numa precisa localização geográfica, em outras palavras, a análise desse tipo de dado permite realizar descobertas baseadas em fatos. Analistas estão frequetemente interessados em observar padrões espaciais e tendências para melhorar seus processos de tomada de decisão. Análise de dados espaciais tem várias aplicações como gerenciamento de cidade inteligentes, gerenciamento de disastres e transporte autônomo \cite{RoddickEHPS04,Telang:2012}.

\section{Problema}

A análise de dados espaciais geralmente é realizada num contexto exploratório: o analista não tem uma consulta precisa em mente e ele explora os dados em passos iterativos a fim de encontrar resultados potencialmente interessantes. Tradicionalmente, um cenário de análise exploratória é descrito na seguinte maneira: o analista visualiza um subconjunto de dados usando uma consulta em ambiente de visualização (por exemplo: Tableau\footnote{\it http://www.tableau.com},
Exhibit\footnote{\it http://www.simile-widgets.org/exhibit/},
Spotfire\footnote{\it http://spotfire.tibco.com}); o resultado será ilustrado em um mapa geográfico; então o analista investiga diferentes partes do conjuto de dados movendo ou focando o mapa afim de encontrar padrões ou tendências de interesse. O analista pode iterar por esse processo várias vezes realizando consultas diferentes e focando em diferentes aspectos.

Contudo, a vasto tamanho do conjunto de dados espacias faz com que o analista se sinta perdido durante a exploração. É possível ter milhares de pontos geográficos em cada bairro de uma cidade, por exemplo. Analistas precisam ter acesso apenas a algumas opções (chamadas de ``highlights'') que ajam como uma direção e assim permitir que ele foque no que lhe interessa na análise. No cenário perfeito, essas opções não são aleatoriamente escolhidas e representam o que o analista se mostrou interessado em iterações passadas.

Neste trabalho, formulamos uma solução para ``realçamento de dados usando feedback coletado ao longo do tempo''. Em outras palavras, buscamos realçar alguns pontos geográficos baseado nos interesses do analista afim de guiá-lo na direção ao que ele deve se concentrar nas iterações seguintes do processo de análise.

\subsection{Caso de Estudo}

Nessa seção, vamos apresentar um caso de estudo afim de demostrar a funcionalidade da nossa abordagem na prática.

\begin{figure}[t]
	\centering
	\includegraphics[width=\textwidth]{imagens/regions}
	\caption{O processo de explorar estadias em Paris.}
	\label{fig:regions}
\end{figure}

{\bf Exemplo.} {\em Lucas está planejando passar alguns dias em Paris, França. Sua apreciação pela cultura francesa faz como que ele tenha interesse em novas experiências na cidade. Ele decidiu por alugar uma estadia pelo Airbnb \footnote{\it http://www.airbnb.com}. Ele gosta de descobrir a cidade, portanto ele é aberto a qualquer tipo de estadia em qualquer região com um leve interesse em ficar perto do centro da cidade. O sistema retorna $4000$ opções diferentes. Como ele não tem outras preferências, uma investigação exaustiva para avaliar cada região da cidade independentemente é necessário, o que é quase impossível. Enquanto estava avaliando algumas opções, ele demostrou interesse na região de  ``Champ de Mars'' (próximo à Torre Eiffel), mas ele esqueceu ou não achou necessário clicar num ponto nessa região. Coletando o feedback do seus movimentos com o mouse no mapa de estadias em Paris, nosso sistema consegue rapidamente detectar o interesse dele na região supracitada e apresentar uma quantidade pequena de opções recomendadas para Lucas.}

Seguimos o exemplo acima para descrever como feedback implícito é coletado na prática. Imagem \ref{fig:regions} mostra os passos de Lucas para explorar estadias em Paris. Imagem \ref{fig:regions}.A mostra os movimentos do mouse dele em diferentes intervalos de tempo. Nesse exemplo, consideramos $g = 3$ e coleta o feedback de Lucas em 3 diferentes intervalos de tempo (evoluindo das Imagens \ref{fig:regions}.B até \ref{fig:regions}.D). Isso mostra que Lucas começou sua busca perto da Torre Eiffel e {\em Arc de Triomphe} (Imagem \ref{fig:regions}.B) e gradualmente mostrou também interesse no sul (Imagem \ref{fig:regions}.C) e norte (Imagem \ref{fig:regions}.D). Todas as interseções entre essas regiões são descobertas (regiões tachadas na Imagem \ref{fig:regions}.E), o que representa um conjunto de Regiões de Denso Interesse (IDR, em inglês, {\em Interesting Dense Regions}), isto é IDR1 até IDR4.

E se Lucas quiser voltar para Paris próximo ano? Ele teria que repetir a mesma anális exploratória, a não ser que ele lembre a localização exata das estadias que ele mostrou no ano passado. Usando nosso sistema, ele não precisaria lembrar, porque suas preferências foram coletadas e poderiam ser usadas para realçar um subconjunto similar ao do ano anterior.

No contexto da análise explorátoria, o analista talvez mude suas preferências entre as sessões (por exemplo, no inverno, Lucas talvez queira ficar próximo ao Torre Eiffel, mas no verão, ele talvez não queira). Afim de atacar esse desafio, também implementamos uma análise temporal para identificar padrões em como as preferências do analistas mudam entre as sessões o que permite nosso método de realçamento ser mais preciso e consistente com o interesse do analista.

\section{Objetivos}

Nessa seção, definimos os objetivos gerais e específicos do nosso trabalho.

\subsection{Objetivos Gerais}

\begin{itemize}
	\item Propor uma abordagem de orientação para exploração de dados espaciais considerando o contexto temporal;
	\item Elaborar como análise temporal pode ser efetivamente aplicada na exploração de dados.
\end{itemize}

\subsection{Objetivos Específicos}

\begin{itemize}
	\item Descrever nosso modelo de dados usado para análise temporal;
	\item Descrever nosso conceito de em Região de Denso Interesse usado para captura de feedback;
	\item Apresentar resultados para nossa abordagem;
\end{itemize}

\section{Organização}

Os próximos capítulos estão organizados na seguinte maneira: no Capítulo \ref{chap:background} discutimos o estado da arte por trás desse trabalho; Capítulo \ref{chap:model} define o modelo de dados; Capítulo \ref{chap:collecting} apresenta como é feito a coleta de feedback durante a análise exploratória; Capítulo \ref{chap:applying} demostra como a análise temporal é aplicada; Capítulo \ref{chap:guiding} apresenta como é realizado o realçamento de pontos de interesse do usuário afim de guiá-lo com base no feedback coletado; Capítulo \ref{chap:experiments} descreve os experimentos e seus resultados; Capítulo \ref{chap:conclusion} conclui e propôe futuros trabalhos.


\chapter{Background}
\label{chap:background}

\section{Related Work}

There are currently several solutions for applying temporal analysis...

\section{GeoGuide}

GeoGuide is an awesome solution for exploring spatial data...

GeoGuide is about making easy to the researcher explore, visualize and learn with spatial datasets.
In order to accomplish a better user experience, we provide a guidence approach based
on his preferences and explicit feedback.
GeoGuide continuously improves its guidance algorithm while
researcher is exploring the dataset. The framework captures implicit feedback
and try to understand what the user wants by analysing how the
researcher preferences has evolved over time.


\chapter{Data Model Definition}
\label{chap:model}

We consider two layers: spatial layer and feedback layer.

{\bf In the spatial layer:} each point in a dataset ($p \in \mathcal{P}$) is described using its coordinates (latitude and longitude) and also associated with a set of attributes ($dom(p)$). For instance, %TODO

{\bf In the feedback layer:} we have IDRs per iteration/session where implicid feedback is captured such mouse moves (or eye gaze). In the beginning, each IDRs is a group of raw points described using its coordinates (latitude and longitude) and a timestamp (the unix timestamp it was captured). These raw points once captured will enter the clustering (for now, ST-DBSCAN) phase to generate the IDR itself with a profile. The profile is built based on the spatial layer and it should represent a summary of its contained points from the spatial layer.

\begin{itemize}
	\item A profile has summary of its spatial points number attributes. For each number attribute in $dom(p)$, we calculate the average, median and standard deviation based on the points contained in the IDR.

	\item A profile has a word rank $R$ of the terms in the text attributes of its spatial points. For each text attribute in $dom(p)$, we evaluate the most used terms in order to create a word rank \cite{kumarAndKaur}.

	\item A profile has a map $M$ between the $<name, value>$ of categoricals attributes and its relevance in $dom(p)$.

	\item TODO: datetime attributes

	\item A profile has a meta property with values such the count of points in the IDR.
\end{itemize}

\chapter{Collecting feedback}
\label{chap:collecting}

TODO

\chapter{Applying temporal analysis}
\label{chap:applying}

TODO

\chapter{Guiding the user}
\label{chap:guiding}

TODO

\chapter{Experiments}
\label{chap:experiments}

TODO

\section{Results}

TODO

\chapter{Conclusion}

To our knowledge...


\section{Contributions}

TODO


\section{Restrictions}

TODO


\section{Future work}

TODO

% Bibliografia (arquivo Capitulos/Referencias.bib)
\bibliography{capitulos/references}
\bibliographystyle{abnt-alf}

% Ap���ndice A (arquivo Includes/ApendiceA)
% \include{capitulos/ApendiceA}

% Anexo A (arquivo Includes/AnexoA)
% \include{capitulos/AnexoA}

% P���gina em branco
\newpage

\end{document}