% Resumo
\begin{center}
	{\Large{\textbf{\myThesis}}}
\end{center}

\vspace{1cm}

\begin{flushright}
	Autor: \myName\\
	Orientador: \mySupervisorName
\end{flushright}

\vspace{1cm}

\begin{center}
	\Large{\textsc{\textbf{Resumo}}}
\end{center}

\noindent Este trabalho propõe um modelo espaço-temporal para exploração de preferências do usuário no cenário da análise exploratória de dados espaciais. A partir da observação das dificuldades enfrentadas por analistas durante a análise exploratória de grandes conjuntos de dados, verificou-se a possibilidade de combinar métodos de exploração de preferências do usuário com análise temporal, o que resultou no modelo de dados proposto neste trabalho. Para tanto, aprimorou-se a ferramenta GeoGuide, um ambiente de exploração de dados espaciais alinhado com métodos de destacamento de informações com base nas preferências do usuário, para coletar dados não somente no contexto espacial, mas também no contexto de domínio. Este trabalho detalha o processo para capturar de maneira transparente as preferências do usuário com o objetivo de viabilizar a aplicação do modelo proposto. Também apresenta como os dados coletados podem serem analisados temporalmente a fim de encontrar padrões nas preferências do usuário. Com este trabalho, espera-se que o modelo proposto seja explorado, adaptado e aplicado além do cenário proposto.

\noindent\textit{Palavras-chave}: Dados Espaciais, Análise Temporal, Preferência do Usuário.