% Resumo em l�ngua estrangeira (em ingl�s Abstract, em espanhol Resumen, em franc�s R�sum�)
\begin{center}
	{\Large{\textbf{\myThesisEnglish}}}
\end{center}

\vspace{1cm}

\begin{flushright}
	Author: \myName\\
	Supervisor: \mySupervisorName
\end{flushright}

\vspace{1cm}

\begin{center}
	\Large{\textsc{\textbf{Abstract}}}
\end{center}

\noindent This work proposes a spatial-temporal model to exploit user preferences in the scenario of an exploratory analysis of spatial data. From the observation of the difficulties faced by analysts during the exploratory analysis of large datasets, the possibility of combining methods of exploring user preferences with time analysis was verified, which resulted in the data model proposed in this work. Therefore, the GeoGuide tool, a spatial data exploration environment which provides a guidance approach using information highlighting methods based on user preferences, has been improved to collect data not only in the spatial context but also in the domain context. This work details the process to transparently capture user preferences with the objective of making the proposed model feasible. It also displays how the data collected can be analyzed temporally in order to find patterns in user preferences. With this work, it is expected that the proposed model will be explored, adapted and applied beyond the presented scenario.

\noindent\textit{Keywords}: Spatial data, Temporal Analysis, User Feedback.