% Resumo em l�ngua estrangeira (em ingl�s Abstract, em espanhol Resumen, em franc�s R�sum�)
\begin{center}
	{\Large{\textbf{Designing and Implementing a Robot That Solves Rubik's Cube Problem Using Arduino}}}
\end{center}

\vspace{1cm}

\begin{flushright}
	Author: Helora Dana Cruz Monteiro\\
	Supervisor: Dr. Plácido Antonio de Souza Neto
\end{flushright}

\vspace{1cm}

\begin{center}
	\Large{\textsc{\textbf{Abstract}}}
\end{center}

\noindent Rubik's cube, known by many as a magic cube, is a three-dimensional puzzle that has a series of steps to reach a solution. Because it is a rather complex puzzle, there are several algorithms composed of sequences of steps to solve it. But to solve the cube in an effective and efficient way, we can use mechanisms that facilitate the process, such as the use of robotics. Robots have been increasingly used to perform many tasks, whether for industry, service delivery or as a way of learning in schools. The cost to design and build a robot is often not accessible by the variety of components it has, so we are always looking for easy and inexpensive ways of developing them in the school environment. One of the proposals that has been adopted for the development of robots is the Arduino platform, which comes as a low cost alternative and easy learning tool where only knowledge in algorithms is needed to develop an application. With this in mind, for a better solution of the problem of the cube and to improve the learning of algorithms a low cost robot with the arduino platform was developed. This paper aims to detail the development process of the robot and its results.

\noindent\textit{Keywords}: Rubik's cube, Arduino.